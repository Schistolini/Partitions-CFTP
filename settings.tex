\documentclass[11pt]{article}
\usepackage{
	graphicx,
	epstopdf,
	fancyhdr,
	amsfonts,
	amsthm,
	amsmath,
	algorithm,
	algorithmic,
	xspace,
	hyperref,
	setspace}
	
\usepackage[left=1in,top=1in,right=1in,bottom=1in]{geometry}
\usepackage{sect sty}	%For centering section headings
\usepackage{enumerate}	%Allows more labeling options for enumerate environments 
\usepackage{epsfig}
\usepackage[space]{grffile}
\usepackage{booktabs}
\usepackage{amsmath}
\usepackage{ amssymb }

\usepackage{minted} % For showing snippets of code
% This will set LaTeX to look for figures in the same directory as the .tex file
%\usemintedstyle{mathematica}
\graphicspath{.} % The dot means current directory.

\pagestyle{fancy}

\lhead{\YOURID}
%\chead{Problem Set \RRNumber }
\rhead{\today}
\lfoot{MATH 457: Partition Theory}
\cfoot{\thepage}
\rfoot{Spring 2024}

% Some commands for changing header and footer format
\renewcommand{\headrulewidth}{0.4pt}
\renewcommand{\headwidth}{\textwidth}
\renewcommand{\footrulewidth}{0.4pt}

\setlength{\parindent}{1.27cm}

\newcommand{\YOURID}{Sam Chistolini}	% Replace "Your Name Here" with your name
\newcommand{\RRNumber}{6}	% Replace 0 with the actual problem set #
\newcommand{\ProblemHeader}	% Don't change this!

% Increase the spacing a bit:
\setstretch{2.0}

\usepackage{titlesec}
\titleformat*{\section}{\LARGE\bfseries}
\titleformat*{\subsection}{\Large\bfseries}
\titleformat*{\subsubsection}{\large\bfseries}
\titleformat*{\paragraph}{\large\bfseries}
\titleformat*{\subparagraph}{\large\bfseries}

\newenvironment{question}[2][Question]{\begin{trivlist}
\item[\hskip \labelsep {\bfseries #1}\hskip \labelsep {\bfseries #2.}]}{\end{trivlist}}

\theoremstyle{definition}
\newtheorem{theorem}{Theorem}[section]
\newtheorem*{theorem*}{Theorem}
\newtheorem{maintheorem}{Theorem}
\renewcommand{\themaintheorem}{\Alph{maintheorem}}
\newtheorem*{maintheorem*}{Main Theorem}
\newtheorem{definition}[theorem]{Definition}
\newtheorem{setting}[theorem]{Setting}
\newtheorem{lemma}[theorem]{Lemma}
\newtheorem{corollary}[theorem]{Corollary}
\newtheorem{proposition}[theorem]{Proposition}
\newtheorem{identity}[theorem]{Identity}
\newtheorem{remark}[theorem]{Remark}
%\newtheorem{question}[theorem]{Question}
\newtheorem{claim}[theorem]{Claim}
\newtheorem{fact}[theorem]{Fact}
\newtheorem{conjecture}[theorem]{Conjecture}
\newtheorem*{note}{Note}
\newtheorem{construction}[theorem]{Construction}
\newtheorem{example}{Example}



\newcommand{\R}{\mathbb{R}}
\newcommand{\N}{\mathbb{N}}
\newcommand{\Q}{\mathbb{Q}}
\newcommand{\Z}{\mathbb{Z}}
\newcommand{\C}{\mathbb{C}}
\newcommand{\LT}{\texttt{LT}}
\newcommand{\LM}{\texttt{LM}}
\newcommand{\mP}{\mathbb{P}}

% It sucks doing the rangle langle stuff so this command does that for you

\newcommand{\ideal}[1]{\ensuremath{\left\langle #1 \right\rangle}}

% For cancellation lines 
\usepackage{cancel}
\newcommand{\bigO}[1]{\ensuremath{\Oc\left( #1 \right)}}

\newcommand{\ve}{\varepsilon}
\usepackage{enumitem}

\usepackage{mathrsfs} % for mathscr
\usepackage{mathtools} % injection symbol

\newcommand{\floor}[1]{\ensuremath{\left\lfloor #1 \right\rfloor}}

 
 % Bijection symbol
 \newcommand{\Oc}{\mathcal{O}}
 \newcommand{\bijectarrow}{%
  \hookrightarrow\mathrel{\mspace{-15mu}}\rightarrow
}

\newcommand{\surjectarrow}{%
  \rightarrow\mathrel{\mspace{-15mu}}\rightarrow
}
 % Roman numerals:
 \newcommand*{\rom}[1]{\expandafter\@slowromancap\romannumeral #1@}
 
\usepackage{ stmaryrd }
\usepackage[style=ieee,maxbibnames=8]{biblatex}


\addbibresource{references.bib}

\usepackage{algpseudocode}
